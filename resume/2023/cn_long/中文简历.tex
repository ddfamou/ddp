% LaTeX file for resume
% This file uses the resume document class (res.cls)

\documentclass{res}[8.5pt]
\usepackage{ctex}
%\usepackage{helvetica} % uses helvetica postscript font (download helvetica.sty)
%\usepackage{newcent}   % uses new century schoolbook postscript font
\newsectionwidth{-3pt}  % So the text is not indented under section headings
\usepackage{fancyhdr}  % use this package to get a 2 line header
\renewcommand{\headrulewidth}{0pt} % suppress line drawn by default by fancyhdr
\setlength{\headheight}{0pt} % allow room for 2-line header
\setlength{\headsep}{0pt}  % space between header and text
\setlength{\headheight}{0pt} % allow room for 2-line header
\usepackage[a4paper,vmargin={10mm,0mm},hmargin={15mm,15mm}]{geometry}
\pagestyle{fancy}     % set pagestyle for document
\rhead{ {\it M. SHAN}\\{\it p. \thepage} } % put text in header (right side)
\cfoot{}                                     % the foot is empty
\topmargin=-0.5in % start text higher on the page
\sectionskip=2ex plus 1ex minus -.2ex % values stolen from LaTeX

\begin{document}
\thispagestyle{empty} % this page has no header
\name{\Large 单梦凡\\[12pt]}% the \\[12pt] adds a blank line after name

\address{{\bf 地址:} 虹口区天宝路466弄, 200086 \hspace{0.25in}  {\bf 电话:} 18017661124 \hspace{0.25in}  {\bf   Email:} dd.famous@gmail.com}

\begin{resume}

\section{{工作经历}}
\vspace{-12pt}
\hrulefill\\
{\bf Block}, 上海 \hfill 2021年5月至今 \\
开发工程师 - MarTech数据团队  \hspace{0.25in} 我2021年加入Afterpay,在2022年初加入Block。
期间主要工作是从头开始设计、搭建和维护MarTech数据平台,包括数据平台基础平台,跨平台数据集成,数据任务,强数据驱动的在线服务,
如多种数据洞察服务,用户数据平台(CDP)等。
同时也负责数据前沿的技术研究,新技术引入,疑难问题处理等。

\vspace{-10pt}
{\bf 上海数禾信息科技有限公司} \hfill 2019年4月  \textasciitilde 2021年4月\\
Hadoop架构师 - 大数据部    \hspace{0.25in} 
成立并负责大数据平台组,
牵头公司大数据平台的架构设计、开发部署、监控管理、优化升级等,负责大数据平台的架构演进,
任职期间从自建小规模大数据平台,主导负责数据上云和云平台迁移,并最终完成数据中台的设计和落地。
与数仓团队和数据应用团队合作,负责公司的数据服务。
同时也负责核心技术研究、疑难问题攻关、开源组件接入,提升平台的运作效率和稳定性。


\vspace{-10pt}
{\bf Works Applications (万革始应用软件有限公司)} \hfill 2016年8月 \textasciitilde 2019年4月\\
团队负责人 – 企业数据处理平台 \hspace{0.25in} 
负责设计、开发大数据开发及运行平台,为业务开发人员提供简便的大数据开发接口,提高业务数据开发效率。平台主要依托 Hadoop/Spark 提供高效低成本的并发方案,并处理不同需求场景中的数据处理请求。

\vspace{-10pt}
\section{{项目经历(近两年)}}
\vspace{-12pt}
\hrulefill\\

\vspace{-25pt}
{\bf 资格服务(Eligibility Service), Block}
我设计并开发了资格服务。资格服务是为下游广告、促销等活动服务提供用户筛选,并为市场营销人员量身定制的数据平台。
它提供基于规则的定位,动态细分,实时响应,以及集成的A/B测试,以提高市场营销的效率,并提高可扩展性和准确性。
同时数据服务于数据分析、机器学习以支持数据驱动的决策和优化。
Kotlin, Armeria, DynamoDB, antlr4,Snowflake

\vspace{-10pt}
{\bf 用户洞察(CustomerIQ), Block}
我设计并开发了用户洞察,一个通用的用户数据平台,包含数据服务,数据管道,夸平台数据集成等功能,旨在为各种场景下的客户属性预测创建一个多功能的系统,
重点为激励/广告观众定位和广告排名/个性化。通过去耦数据管道和消费者服务,实现独立添加属性/特点,在不依赖于其他团队的情况下,优化市场营销的效果。
Kotlin, DynamoDB, Snowflake, Airflow

\vspace{-10pt}
{\bf 数据洞察 (Insight Service), Block}
设计、部署并落地了多个数据洞察服务,包括数据平台基础设施,数据流,数据质量系统和数据服务。
数据洞察是一个自助的数据服务平台,是我建数据闭环归的关键,
将商户数据整合后提供个各地商户,并提供各种对比趋势,集成AI分析提供数据分析和营销建议。
目前服务超过6000个各地商户,并帮助商家做出数据驱动的决策。
Clickhouse, Kylin, Spark, Hive, DynamoDB,OpenAI

\vspace{-10pt}
{\bf 数据中台平台,数禾科技}
作为大数据平台负责人,设计并主导了大数据平台的整体迭代。
大数据平台从最初的小规模自建,随着数据量增长逐步数据上云,又因云供应商切换进行云迁移,最终在完成数据中台。
保证在数据迁移和平台重构的过程中对业务无影响,同时修复发现开源组件的问题(HDFS-9406, HIVE-19994),并减少大数据平台整体的运行成本。
Cloudera, AWS EMR, S3, Ali EMR, OSS, Dataphin

\vspace{-10pt}
{\bf 特征平台,数禾科技}
特征平台用于为模型训练及模型预测提供所需特征。传统的特征平台主要聚焦于数据清洗、归一化、分箱等数据处理阶段,
但对于数据获取并没有很好的解决方案。而在实际的模型训练到模型部署上线预测的 过程中,最困难的问题就在于数据获取。
因此,本系统直接聚焦于这一点,所有设计将为同一点服务:离线训练 特征与实时预测特征统一
HBase, Flink, Hive


\vspace{-10pt}
\section{常用技能}
\vspace{-12pt}
\hrulefill\\
{\sl 语言:}  Java,Kotlin,Python,Ansible,shell\\
{\sl 组件:} Clickhouse, Kylin, Hive, HBase, Hadoop, Spark, Flume, Flink, Zookeeper, Kafka, neo4j, HugeGraph\\
{\sl 技能:} AWS,阿里云,Snowflake

\vspace{-12pt}
\section{{教育背景}}
\vspace{-12pt}
\hrulefill\\
{\bf 南京大学} \hfill  2010年9月\textasciitilde 2016年6月 \hspace{0.25in} 
{\sl 博士}, 计算机科学与技术系\\
{\bf 明尼苏达大学双城分校} \hfill 2014年10月\textasciitilde 2015年10月 \hspace{0.25in} 
{\sl 访学}, 计算机科学与工程系  \\
{\bf 南京大学}  \hfill 2006年9月\textasciitilde 2010年6月  \hspace{0.25in} 
{\sl 学士}, 计算机科学与技术系

\vspace{-12pt}
\section{{代码库}}
\vspace{-12pt}
\hrulefill\\
{\bf GitHub} \hfill https://github.com/ddfamou

\end{resume}
\end{document}
